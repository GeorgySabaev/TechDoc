% creator: ilyakooo0
% extra credit: dashared
% version: 1.0.1

\PassOptionsToPackage{main=russian,english}{babel}

\documentclass{../TechDoc}
\usepackage{enumitem}

\title{3D Renderer с Нуля}
\author{Студент группы БПИ 216}{Г. А. Сабаев}
\academicTeacher{Доцент департамента больших данных и информационного поиска факультета компьютерных наук}{Е. О. Кантонистова}

\documentTitle{Техническое задание}
\documentCode{RU.17701729.05.01-01 ТЗ 01-1}

\begin{document}
    \maketitle
    
    \tableofcontents
    
    \section{Введение}

\subsection{Наименование программы}

\subsubsection{Наименование программы на русском языке}

Библиотека для 3D-рендеринга <<cpp\_rend>>

\subsubsection{Наименование программы на английском языке}

<<cpp\_rend>> 3D-rendering library.

\subsection{Краткая характеристика области применения}

Стандартной практикой для приложений, работающих с 3D, является использование видеокарты. Такой подход имеет, как главное преимущество, исключительно быструю скорость работы, что крайне важно для крайне ресурсоемких приложений, отрисовывающих десятки кадров в секунду в режиме реального времени.

Популярные драйверы для видеокарт поддерживают API, такие как OpenGL или Vulkan, которые заключаются в передаче видеокарте шейдерного кода, который будет скомпилирован и выполнен самой видеокартой. Этот подход увеличивает производительность и предоставляет высокоуровневый фреймворк пользователю API, но также делает модификацию основных алгоритмов рендеринга затруднительной.

Библиотека для 3D-рендеринга <<cpp\_rend>> (далее - cpp\_rend) не использует видеокарту для рендеринга, открывая весь алгоритм для расширения и модификации конечным пользователем взамен на резкую потерю производительности в случаях когда она является менее критичной, и без значительных потерь на устройствах без видеокарты.

\subsection{Основание для разработки}

Основанием для разработки является учебный план подготовки бакалавров по направлению 09.03.04 «Программная инженерия» и утвержденная академическим руководителем тема курсового проекта.

    
    \section{Назначение разработки}

\subsection{Функциональное назначение}

Функциональным назначением <<Renderer>> является отрисовка интерактивной 3D сцены на экране с точки зрения находящейся в этой сцене камеры.

<<Renderer>> должен состоять из двух частей - библиотека с реализацией пайплайна 3D-отрисовки и интерактивное приложение.
Библиотека должна предоставлять методы и классы для создания, модификации и отрисовки 3D сцены. 
Приложение должно выводить на экран изображение сцены и выполнять движение и поворот камеры в пространстве в ответ на пользовательский ввод.

\subsection{Эксплуатационное назначение}

Данное приложение может быть использовано в образовательных целях, для изучения внутреннего устройства алгоритмов и техник 3D рендеринга, лежащих в основе методов, которые реализованы в графических ускорителях.


    \section{Требования к программе}
    
    \subsection{Требования к функциональным характеристикам}
    \subsubsection{Состав выполняемых функций}
    
    \begin{enumerate}
    \item Программа должна разделяться на библиотеку и интерактивное приложение;
    \item Библиотека должна предоставлять класс сцены;
    \item Сцена должна хранить набор (возможно пустой) источников направленного света;
    \item Источник направленного света должен хранить направление и интенсивность;
    \item Сцена должна хранить набор (возможно пустой) объектов сцены, называемых корневыми;
    \item Объект сцены должен хранить набор (возможно пустой) дочерних объектов;
    \item Объект сцены должен хранить свое положение в пространстве относительно объекта-предка для которого он является дочерним (глобальные координаты если объект корневой);
    \item Матрицы позиций всех объектов сцены в глобальных координатах должны кешироваться до следующей модификации;
    \item Объект должен хранить точечный источник света (если он указан для данного объекта).
    \item Точечный источник света должен хранить дальность и интенсивность;
    \item Объект должен хранить 3D-модель (если она указана для данного объекта).
    \item 3D-модель должна хранить набор треугольных полигонов.
    \item Полигон должен хранить 3 точки с вещественными координатами, представляющими собой его вершины.
    \item Полигон должен хранить 3 нормальных вектора, по одному для каждой его вершины.
    \item 3D-модель должна хранить шейдер.
    \item Библиотека должна предоставлять пользователю возможность создавать собственные шейдеры как пользовательские классы;
    \item Библиотека должна предоставлять пользователю класс, реализующий минималистичный шейдер по модели Блинна-Фонга;
    \item Библиотека должна предоставлять интерфейс работы с треугольным полигоном:
    \begin{enumerate}
        \item возможность создания полигона;
        \item возможность задания нормальных векторов в каждой вершине полигона;
        \item возможность добавления полигона к модели;
    \end{enumerate}
    \item Библиотека должна предоставлять интерфейс работы с 3D моделью:
    \begin{enumerate}
        \item возможность создания модели;
        \item возможность задания шейдера модели;
        \item возможность добавления полигона к модели;
        \item возможность загрузки модели из файла формата .obj;
        \item возможность изменения данных модели напрямую с помощью кода;
    \end{enumerate}	
     \item Библиотека должна предоставлять интерфейс работы с объектом:
    \begin{enumerate}
        \item возможность изменения относительного положения объекта в пространстве;
        \item возможность добавления 3D-модели к объекту;
        \item возможность добавления точечного источникa света к объекту;
    \end{enumerate}
    \item Библиотека должна предоставлять интерфейс работы со сценой:
    \begin{enumerate}
        \item возможность создания сцены;
        \item возможность добавления источника направленного света;
        \item возможность добавления корневого объекта;
        \item возможность удаления источника направленного света;
        \item возможность удаления корневого объекта;
        \item возможность итерирования по иерархии объектов сцены;
    \end{enumerate}
    \item Библиотека должна предоставлять класс, представляющий собой камеру.
    \item Камера должна хранить разрешение экрана.
    \item Камера должна хранить вертикальный угол обзора.
    \item Камера должна хранить свои координаты в глобальном пространстве.
    \item Библиотека должна предоставлять интерфейс работы с камерой:
    \begin{enumerate}
        \item возможность задания разрешения;
        \item возможность задания угла обзора (field of view);
        \item возможность задания положения в пространстве;
        \item функцию перевода точки из глобальных координат в NDC (Normalized Device Coordinates) конкретной камеры;
    \end{enumerate}
    \item Библиотека должна предоставлять класс-функтор рендерера, который должен:
    \begin{enumerate}
        \item принимать сцену и камеру;
        \item переводить все объекты в NDC с помощью камеры;
        \item для каждого пикселя определять ближайший к камере полигон (если он есть) путем отрисовки всех полигонов с помощью z-буфера.
        \item для каждого пикселя вычислять информацию о его позиции на полигоне (uv-координаты).
        \item для каждого пикселя вычислять информацию о его освещенности.
        \item передавать информацию о каждом пикселе и данных о нем в соответствующий шейдер;
        \item окрашивать каждый пиксель изображения в цвет, возвращаемый шейдером;
        \item возвращать изображение сцены с точки зрения камеры в формате RGBA32;
    \end{enumerate}
    \item Класс-функтор рендерера должен хранить все внутренние буферы как внутренние поля (во избежание множества выделений памяти каждый кадр);
    \item Приложение должно выводить на экран окно;
    \item Приложение должно заполнять внутреннее пространство окна изображением-результатом рендеринга сцены с точки зрения камеры;
    \item Приложение должно предоставлять возможность пользователю управлять вращением камеры посредством захвата курсора мыши;
    \item Приложение должно предоставлять возможность пользователю передвигать камеру с помощью клавиш клавиатуры;
    \item Приложение должно обновлять изображение сцены на экране в соответствии с движением камеры и/или объектов сцены.
    \item Приложение должно использовать маскимально полный набор возможностей библиотеки (включая реализованный доп. функционал) для наилучшей демонстрации реализованных функций.
\end{enumerate}

\subsubsection{Дополнительный функционал}

В качестве дополнительного функционала программы выступает:
\begin{enumerate}
    \item возможность рендеринга в HDR (High Dynamic Range);
    \item возможность перевода изображения из HDR в RGBA32 с помощью приближенной реализации ACES за авторством Krzysztof Narkowicz;
    \item возможность применения эффекта Bloom к изображению;
    \item возможность загрузки произвольной сцены из файла;
\end{enumerate}

        
    \subsubsection{Организация входных данных}

    Приложение в качестве входных данных получает нажатия клавиш клавиатуры пользователя и движения курсора.

    \subsubsection{Организации выходных данных}
    
    Приложение в качестве выходных данных выводит на экран изображение сцены с точки зрения камеры пользователя.

    \subsection{Требования к временным характеристикам}

    Требований к временным характеристикам нет.

    \subsection{Требования к интерфейсу}
        
    Классы и методы, предоставляемые библиотекой, описаны в секции "Состав выполняемых функций"

Интерфейс приложения представляет собой одно окно, заполненное изображением сцены. При нажатии на окно программа прячет курсор и перехватывает движения мыши для управления вращения камеры.

    \subsection{Требования к надежности}
        
    \subsubsection{Требования к обеспечению надежного (устойчивого) функционирования программы}
Требования к обеспечению надежного (устойчивого) функционирования программы совпадают с требованиями к обеспечению надежного (устойчивого) функционирования программы устройства, на котором выполняется программа.

\subsubsection{Время восстановления после отказа}
Время восстановления после отказа, вызванного сбоем электропитания или не фатальным сбоем операционной системы, не должно превышать времени, необходимого на перезагрузку ОС и запуск программы.
Время восстановления после отказа, вызванного неисправностью технических средств или операционной системы не должно превышать времени, требуемого на устранение неисправностей. 

\subsubsection{Отказы из-за некорректных действий оператора}
Отказы программы возможны вследствие некорректных действий оператора (пользователя) при взаимодействии с операционной системой. Во избежание возникновения отказов программы по указанной выше причине следует обеспечить работу конечного пользователя без предоставления ему привилегий администратора.

    \subsection{Условия эксплуатации}
    
    \subsubsection{Климатические условия}
    
    Климатические условия сопадают с климатическими условиями эксплуатации устройства.

    \subsubsection{Требования к пользователю}
    
    Пользователь приложения должен иметь понимание того как работать с операционной системой и устройством на которых запущена программа.

    \subsection{Требования к составу и параметрам технических средств}
    
    Для корректной работы программы необходимо устройство с не менее чем 2 ГБ оперативной памяти и наличием дисплея, мыши и клавиатуры.
    При условии выполнения Требований к информационной и программной совместимости других ограничений не присутствует; однако для повышения частоты кадров рекомендуется устройство с процессором, по производительности эквивалентном или превосходящем Intel® Core™ i7-5500U.
    
    \subsection{Требования к информационной и программной совместимости}
    
    Приложение должно быть написано на языке C++ версии C++20, успешно компилироваться с помощью компиляторов MSVC и GCC при условии наличия CMake версии $\geq$ 3.28.1, Conan2 версии $\geq$  2.0.17, корректной работы библиотек Eigen3 версии$ \geq$  3.4.0 и SFML версии $\geq$  2.6.1 для целевой платформы. 
    Приложение должно использовать clang-tidy версии $\geq$ 10, clang-format версии $\geq$ 10.
    
    \subsection{Требования к составу сетевых средств}
    
    У устройства должен быть доступ к сети интернет для скачивания и установки приложения.

    \subsection{Требования к маркировке и упаковке}
     Программа распространяется в виде электронного пакета, содержащего программную документацию и приложение.
    
    \subsection{Требования к транспортировке и хранению}
    Приложение должно размещаться на платформе GitHub и иметь возможность быть загруженной оттуда потенциальным пользователем.

    \section{Требования к программной документации}

    \input{includes/docs}


    \section{Технико-экономические показатели}
     \subsection{Ориентировочная экономическая эффективность}
    	Экономическая эффективность в рамках курсового проекта не предусмотрена.
     \subsection{Предполагаемая потребность}
          Данный продукт могут использовать разработчики во время изучения 3D-рендеринга.
     \subsection{Экономические преимущества разработки по сравнению с отечественными и зарубежными аналогами}
          В рамках данного задания экономические преимущества по сравнения с отечественными и зарубежными аналогами не предусмотрена.

    \section{Стадии и этапы разработки}
    
    \begin{enumerate}
        \item техническое задание:
        \begin{enumerate}
            \item этапы разработки:
            \begin{enumerate}
                \item обоснование необходимости разработки программы; 
                \item постановка задачи; 
                \item сбор исходных материалов; 
                \item выбор и обоснование критериев эффективности и качества разрабатываемой программы; 
                \item обоснование необходимости проведения научно-исследовательских работ; 
            \end{enumerate}
            \item разработка и утверждение технического задания:
            \begin{enumerate}
                \item определение требований к программе; 
                \item определение стадий, этапов и сроков разработки программы и документации на неё; 
                \item согласование и утверждение технического задания; 
            \end{enumerate}
        \end{enumerate}
    \item технический проект:
    \begin{enumerate}
        \item разработка технического проекта:
        \begin{enumerate}
            \item уточнение структуры входных и выходных данных; 
            \item разработка алгоритма решения задачи; 
            \item определение формы представления входных и выходных данных; 
            \item разработка структуры программы; 
            \item окончательное определение конфигурации технических средств. 
        \end{enumerate}
        \item утверждение технического проекта:
        \begin{enumerate}
            \item разработка пояснительной записки; 
            \item согласование и утверждение технического проекта. 
        \end{enumerate}
    \end{enumerate}
    \item рабочий проект:
    \begin{enumerate}
        \item разработка программы:
        \begin{enumerate}
            \item программирование и отладка программы. 
        \end{enumerate}
        \item разработка программной документации:
        \begin{enumerate}
            \item разработка программных документов в соответствии с требованиями гост 19.101-77. 
        \end{enumerate}
        \item испытания программы:
        \begin{enumerate}
            \item разработка, согласование и утверждение порядка и методики испытаний; 
            \item корректировка программы и программной документации по результатам испытаний.
        \end{enumerate}
    \end{enumerate}
    \end{enumerate}

    \section{Порядок контроля и приемки}
    
    Контроль и приемка разработки осуществляются в соответствии с документом «Программа и методика испытаний».
    
    \registrationList
        
\end{document}