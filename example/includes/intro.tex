\section{Введение}

\subsection{Наименование программы}

\subsubsection{Наименование программы на русском языке}

Библиотека для 3D-рендеринга <<cpp\_rend>>

\subsubsection{Наименование программы на английском языке}

<<cpp\_rend>> 3D-rendering library.

\subsection{Краткая характеристика области применения}

Стандартной практикой для приложений, работающих с 3D, является использование видеокарты. Такой подход имеет, как главное преимущество, исключительно быструю скорость работы, что крайне важно для крайне ресурсоемких приложений, отрисовывающих десятки кадров в секунду в режиме реального времени.

Популярные драйверы для видеокарт поддерживают API, такие как OpenGL или Vulkan, которые заключаются в передаче видеокарте шейдерного кода, который будет скомпилирован и выполнен самой видеокартой. Этот подход увеличивает производительность и предоставляет высокоуровневый фреймворк пользователю API, но также делает модификацию основных алгоритмов рендеринга затруднительной.

Библиотека для 3D-рендеринга <<cpp\_rend>> (далее - cpp\_rend) не использует видеокарту для рендеринга, открывая весь алгоритм для расширения и модификации конечным пользователем взамен на резкую потерю производительности в случаях когда она является менее критичной, и без значительных потерь на устройствах без видеокарты.

\subsection{Основание для разработки}

Основанием для разработки является учебный план подготовки бакалавров по направлению 09.03.04 «Программная инженерия» и утвержденная академическим руководителем тема курсового проекта.
