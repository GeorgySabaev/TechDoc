\section{Введение}

\subsection{Наименование программы}

\subsubsection{Наименование программы на русском языке}

<<3D Renderer с Нуля>>

\subsubsection{Наименование программы на английском языке}

<<3D Renderer from Scratch>>

\subsubsection{Краткое наименование программы}

<<Renderer>>

\subsection{Краткая характеристика области применения}
Стандартной практикой для приложений, работающих с 3D, является использование видеокарты (GPU). Такой подход имеет, как главное преимущество, исключительно быструю скорость работы, что крайне важно для крайне ресурсоемких приложений, отрисовывающих десятки кадров в секунду в режиме реального времени.

Популярные драйверы для GPU поддерживают API, такие как OpenGL или Vulkan, которые заключаются в передаче видеокарте шейдерного кода, который будет скомпилирован и выполнен самой видеокартой. Этот подход увеличивает производительность и предоставляет высокоуровневый фреймворк пользователю API, но 
скрывает множество низкоуровневых деталей пайплайна отрисовки.

<<Renderer>> - программа для отрисовки 3D-сцен на монитор компьютера, реализующая все вычисления на ЦПУ. Разработка программы производится для образовательных целей, и весь процесс отрисовки изображения реализуется с нуля в программном коде. Такой подход позволяет разобраться в относительно низкоуровневых процессах и алгоритмах 3D-рендеринга, ктороые обычно скрыты для пользователя.

\section{Основание для разработки}

Основанием для разработки является учебный план подготовки бакалавров по направлению 09.03.04 «Программная инженерия» и утвержденная академическим руководителем тема курсового проекта.
