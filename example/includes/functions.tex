Библиотека должна предоставлять пользователю возможность собирать и модифицировать сцену, содержащую иерархию объектов в 3d-пространстве, некоторые из которых имеют 3d-модели из треугольных полигонов, и производить рендер в 2D-изображение с учетом перспективы и заданной позиции камеры с помощью матриц проективного предоразования пространства и z-буфера.

Библиотека должна сопровождаться интерактивным демо-приложением (далее - демо), демонстрирующим функционал библиотеки и являющемся прикладным примером ее использования.

Разрабатываемая библиотека должна: 
\begin{enumerate}
    \item предоставлять интерфейс работы со сценой:
    \begin{enumerate}
        \item возможность создания сцены;
        \item возможность задания параметров глобального освещения сцены;
        \item возможность добавления объекта в иерархию объектов сцены;
        \item возможность итерирования по иерархии объектов сцены;
        \item возможность изменения положения объекта иерархии в пространстве;
        \item возможность добавления 3D-модели к объекту иерархии;
        \item возможность добавления шейдера к объекту иерархии;
    \end{enumerate}
    \item кешировать матрицы позиций объектов сцены в глобальных координатах;
    \item предоставлять пользователю возможность создавать собственные шейдеры как пользовательские классы;
    \item предоставлять пользователю класс, реализующий минималистичный шейдер по модели Блинна-Фонга;
   \item предоставлять интерфейс работы со 3D-моделями:
    \begin{enumerate}
        \item возможность загрузки модели из файла формата .obj;
        \item возможность изменения данных модели напрямую с помощью кода;
    \end{enumerate}
    \item предоставлять интерфейс работы с камерой:
    \begin{enumerate}
        \item возможность задания разрешения;
        \item возможность задания угла обзора (field of view);
        \item возможность задания положения в пространстве;
        \item функцию перевода точки из глобальных координат в NDC (Normalized Device Coordinates) конкретной камеры;
    \end{enumerate}
    \item предоставлять класс-функтор рендерера, который должен:
    \begin{enumerate}
        \item принимать сцену и камеру;
        \item переводить все объекты в NDC с помощью камеры;
        \item для каждого пикселя определять ближайший к камере полигон (если он есть) путем отрисовки всех полигонов с помощью z-буфера.
        \item для каждого пикселя получать информацию о его позиции на полигоне (uv-координаты).
        \item для каждого пикселя получать информацию о его освещенности.
        \item передавать информацию о каждом пикселе и данных о нем в соответствующий шейдер;
        \item окрашивать каждый пиксель изображения в цвет, возвращаемый шейдером;
        \item возвращать изображение сцены с точки зрения камеры в формате RGBA32;
    \end{enumerate}
    \item хранить все внутренние буферы рендерера как поля класса (во избежание множества выделений памяти каждый кадр);
\end{enumerate}

Разрабатываемое демо должно: 
\begin{enumerate}[resume]
    \item выводить на экран результат рендеринга сцены с точки зрения камеры;
    \item предоставлять возможность пользователю управлять вращением камеры посредством захвата курсора мыши;
    \item предоставлять возможность пользователю передвигать камеру с помощью клавиш клавиатуры;
    \item обновлять изображение сцены на экране в соответствии с движением камеры и/или объектов сцены.
    \item использовать маскимально полный набор возможностей библиотеки (включая реализованный доп. функционал) для наилучшей демонстрации.
\end{enumerate}

В качестве дополнительного функционала библиотеки выступает:
\begin{enumerate}[resume]
    \item возможность создания точечных источников света;
    \item возможность рендеринга в HDR (High Dynamic Range);
    \item возможность перевода изображения из HDR в RGBA32 с помощью приближенной реализации ACES за авторством Krzysztof Narkowicz;
    \item возможность применения эффекта Bloom к изображению;
\end{enumerate}

В качестве дополнительного функционала демо выступает:
\begin{enumerate}[resume]
    \item возможность загрузки произвольной сцены из файла;
\end{enumerate}
