\begin{enumerate}
    \item Программа должна разделяться на библиотеку и интерактивное приложение;
    \item Библиотека должна предоставлять класс сцены;
    \item Сцена должна хранить набор (возможно пустой) источников направленного света;
    \item Источник направленного света должен хранить направление и интенсивность;
    \item Сцена должна хранить набор (возможно пустой) объектов сцены, называемых корневыми;
    \item Объект сцены должен хранить набор (возможно пустой) дочерних объектов;
    \item Объект сцены должен хранить свое положение в пространстве относительно объекта-предка для которого он является дочерним (глобальные координаты если объект корневой);
    \item Матрицы позиций всех объектов сцены в глобальных координатах должны кешироваться до следующей модификации;
    \item Объект должен хранить точечный источник света (если он указан для данного объекта).
    \item Точечный источник света должен хранить дальность и интенсивность;
    \item Объект должен хранить 3D-модель (если она указана для данного объекта).
    \item 3D-модель должна хранить набор треугольных полигонов.
    \item Полигон должен хранить 3 точки с вещественными координатами, представляющими собой его вершины.
    \item Полигон должен хранить 3 нормальных вектора, по одному для каждой его вершины.
    \item 3D-модель должна хранить шейдер.
    \item Библиотека должна предоставлять пользователю возможность создавать собственные шейдеры как пользовательские классы;
    \item Библиотека должна предоставлять пользователю класс, реализующий минималистичный шейдер по модели Блинна-Фонга;
    \item Библиотека должна предоставлять интерфейс работы с треугольным полигоном:
    \begin{enumerate}
        \item возможность создания полигона;
        \item возможность задания нормальных векторов в каждой вершине полигона;
        \item возможность добавления полигона к модели;
    \end{enumerate}
    \item Библиотека должна предоставлять интерфейс работы с 3D моделью:
    \begin{enumerate}
        \item возможность создания модели;
        \item возможность задания шейдера модели;
        \item возможность добавления полигона к модели;
        \item возможность загрузки модели из файла формата .obj;
        \item возможность изменения данных модели напрямую с помощью кода;
    \end{enumerate}	
     \item Библиотека должна предоставлять интерфейс работы с объектом:
    \begin{enumerate}
        \item возможность изменения относительного положения объекта в пространстве;
        \item возможность добавления 3D-модели к объекту;
        \item возможность добавления точечного источникa света к объекту;
    \end{enumerate}
    \item Библиотека должна предоставлять интерфейс работы со сценой:
    \begin{enumerate}
        \item возможность создания сцены;
        \item возможность добавления источника направленного света;
        \item возможность добавления корневого объекта;
        \item возможность удаления источника направленного света;
        \item возможность удаления корневого объекта;
        \item возможность итерирования по иерархии объектов сцены;
    \end{enumerate}
    \item Библиотека должна предоставлять класс, представляющий собой камеру.
    \item Камера должна хранить разрешение экрана.
    \item Камера должна хранить вертикальный угол обзора.
    \item Камера должна хранить свои координаты в глобальном пространстве.
    \item Библиотека должна предоставлять интерфейс работы с камерой:
    \begin{enumerate}
        \item возможность задания разрешения;
        \item возможность задания угла обзора (field of view);
        \item возможность задания положения в пространстве;
        \item функцию перевода точки из глобальных координат в NDC (Normalized Device Coordinates) конкретной камеры;
    \end{enumerate}
    \item Библиотека должна предоставлять класс-функтор рендерера, который должен:
    \begin{enumerate}
        \item принимать сцену и камеру;
        \item переводить все объекты в NDC с помощью камеры;
        \item для каждого пикселя определять ближайший к камере полигон (если он есть) путем отрисовки всех полигонов с помощью z-буфера.
        \item для каждого пикселя вычислять информацию о его позиции на полигоне (uv-координаты).
        \item для каждого пикселя вычислять информацию о его освещенности.
        \item передавать информацию о каждом пикселе и данных о нем в соответствующий шейдер;
        \item окрашивать каждый пиксель изображения в цвет, возвращаемый шейдером;
        \item возвращать изображение сцены с точки зрения камеры в формате RGBA32;
    \end{enumerate}
    \item Класс-функтор рендерера должен хранить все внутренние буферы как внутренние поля (во избежание множества выделений памяти каждый кадр);
    \item Приложение должно выводить на экран окно;
    \item Приложение должно заполнять внутреннее пространство окна изображением-результатом рендеринга сцены с точки зрения камеры;
    \item Приложение должно предоставлять возможность пользователю управлять вращением камеры посредством захвата курсора мыши;
    \item Приложение должно предоставлять возможность пользователю передвигать камеру с помощью клавиш клавиатуры;
    \item Приложение должно обновлять изображение сцены на экране в соответствии с движением камеры и/или объектов сцены.
    \item Приложение должно использовать маскимально полный набор возможностей библиотеки (включая реализованный доп. функционал) для наилучшей демонстрации реализованных функций.
\end{enumerate}

\subsubsection{Дополнительный функционал}

В качестве дополнительного функционала программы выступает:
\begin{enumerate}
    \item возможность рендеринга в HDR (High Dynamic Range);
    \item возможность перевода изображения из HDR в RGBA32 с помощью приближенной реализации ACES за авторством Krzysztof Narkowicz;
    \item возможность применения эффекта Bloom к изображению;
    \item возможность загрузки произвольной сцены из файла;
\end{enumerate}
